\chapter{Bevezetés}
Ahogyan a mérnöki tudományok egyre szerteágazóbbak, a tervezett rendszerek pedig egyre összetettebbek lettek --~mind a komponensek, mind az érintett szakterületek számának tekintetében~--, megjelent az igény, hogy magukat a rendszereket is külön szakemberek tervezzék, ahogy az egyes alkotórészeket is~\cite{Gianni2017}.
Így született meg a rendszertervezés területe, melynek elsődleges feladata a rendszerrel szemben támasztott követelmények azonosítása és teljesülésének garantálása.

Kezdetben ezeket a feladatokat írásos dokumentumok segítségével készítették a rendszermérnökök, akik ezeknek a dokumentumoknak a naprakészen tartásával és az ezek közti hivatkozásokkal garantálták a tervek konzisztenciáját és a rendszer működőképességét.
Ahogy a rendszerek komplexitása tovább emelkedett, ez a megközelítés kezdett tarthatatlanná válni. Egyre gyakoribbá váltak a tervezési hibák és egyre nagyobb terhet jelentett a mérnökök számára a rendszer állapotát tükröző dokumentumok naprakészen tartása.
Erre a problémára nyújtott megoldást a modellalapú rendszertervezés megjelenése, ahol a folyamat központi eleme egy matematikailag is formalizált rendszermodell, amely a számítógépek számára is értelmezhető, ezzel meggyorsítva, egyszerűsítve és gyakran automatizálva olyan feladatokat, melyek korábban komoly terhet jelentettek a rendszermérnököknek és számos hibalehetőséget okoztak a tervezési folyamatban.

Mára a terület nagyon sokat fejlődött és szorosan összefonódott a kritikus rendszerek --~mint például a repülőgépek, atomerőművek, vagy vasúti biztosítóberendezések~-- tervezésével. Ennek elsődleges oka, hogy ezeknek a rendszereknek a fejlesztése során elengedhetetlen, hogy minden egyes követelmény teljesüljön és nyomonkövethető legyen.
Azért fontos ezzel külön rendszerszinten is foglalkozni, mert tökéletesen működő alkatrészekből is lehet hibás rendszert építeni, mivel az nem egyszerűen alkatrészek összessége, működését az egyes komponensek megfelelő együttműködése garantálja.

Nem meglepő tehát, hogy a rendszertervezés területén is megjelentek a különböző validációs és verifikációs technikák, melyeknek célja, hogy a megfelelő rendszert és a megfelelő módon tervezzük meg.

Az utóbbi években egyre több figyelmet kapott a rendszertervezők részéről a szimulációs technológiák, mint validációs és verifikációs eszköz felhasználása a rendszertervezés területén~\cite{Ma_2022}.
Ezeket a technikákat már évtizedek óta használják ilyen célra más műszaki területeken, ezentúl számos esetben sikeresen használták fel őket a fejlesztés felgyorsítására és gazdaságosabbá tételére is.

Ennek hatására számos munka született az elmúlt években a szimulációk potenciális előnyeiről a rendszertervezés területén és számos további lehetőséget azonosítottak, melyek a szimulációs technikák és a modellalapú rendszertervezés integrációjával további előnyöket biztosítanának a rendszermérnököknek.
Mindennek ellenére a két területek integrációjára még nem született széles körben elterjedt, szabványos megoldás.

Ezzel párhuzamosan a szabványosítás utolsó fázisába lépett a modellalapú rendszertervezésben legelterjedtebb SysML nyelv teljesen új, korszerűsített változata a SysML v2~\cite{Bajaj_2022}.
Az új nyelv nagyon sokat fejlődött az elődjével kapcsolatos tapasztalatokból. A teljesen új, rendszertervezésre szabott metamodellen\footnote{A nyelv belső szerkezetét és működését leíró modell.} túl számos új elemmel bővült melyeknek célja a külső források kezelése, a követelmények nyomonkövethetőségének további javítása és a fejlesztőkörnyezetek és egyéb automatikus eszköztámogató szoftverek felé biztosított szabványos interfészek.

Ehhez a merőben új nyelvhez új fejlesztési módszertanok kidolgozására is szükség van.
A SysML-hez készült módszertanok esetében többről bejelentették már, hogy készül a SysML v2-re átültetett verzió, azonban ezek sajnos még nem használják majd ki az új szabvány által nyújtott gazdag új eszköztárat.
Azonban hosszú távon olyan módszerekre van szükség, amelyek kihasználják ezeket a lehetőségeket, hiszen különben a bennük rejlő lehetőségek kiaknázatlanok maradnak. Persze ehhez a módszertanokon túl eszköztámogatásra is szükség lesz, de először a modellezési módszerek területén kell megalapozni ezeket a további fejlesztéseket.

Ennek a dolgozatnak a célkitűzése egy ilyen, az új nyelvi elemeket kihasználó módszertan alapjainak kidolgozása.
További cél, hogy a javasolt módszereket kezdetettől fogva a szimulációs technológiák figyelembevételével, azokkal összhangban legyenek kidolgozva.
Ez hosszútávon megoldást jelenthetne a fent említett integrációs problémákra. Ennek a célnak az elérése jóval könnyebb a SysML v2 nyújtotta új módszerek segítségével.
További segítséget nyújt továbbá, hogy egy alapoktól szimulációkkal együtt tervezett módszertan kidolgozása jóval egyszerűbb feladat, mint összhangba hozni olyan módszereket a technológiával, amelyek kidolgozása során még nem volt szempont egy ilyen lehetőség kihasználása.

Ez a dolgozat egy ilyen fejlesztési módszertannal szembeni követelmények és a megvalósítás lehetséges irányait gyűjti össze, majd értékeli ki.

Ezután bemutatok egy az összegyűjtött adatokra támaszkodva megtervezett SysML v2-t használó módszertan alapjait, amely a rendszermodellek új szabvánnyal való tervezésén túl a szimulációs technológiák integrálására is alkalmas.

A kidolgozott módszertan javaslatot ezután egy esettanulmány keretein belül egy kvadkopter rendszermodelljének és szimulációs modelljeinek elkészítésére használom fel, majd ezen keresztül mutatom be az így gyűjtött tapasztalatokat, valamint a módszerek nagyobb előnyeinek a gyakorlatban való megjelenését.

A dolgozat végén értékelem a módszertan kidolgozott alapjait, a tervezési folyamatban nyújtott előnyeik, valamint a rendszertervezés, szimuláció és ezek integrációja kapcsán elért eredményeit.

Azokban az esetekben, ahol a javasolt módszerek csupán előkészítik további eszközök használatát, és csak az ezekkel való kompatibilitást valósítják meg, ott összefoglalom, hogy milyen további kutatási és fejlesztési irányok szükségesek a két terület integrációjának megvalósításához.